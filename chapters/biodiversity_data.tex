\chapter{Understanding Biodiversity Occurrence Data}\label{biodiversity_data}

% ==============
% Herbarium data
% --------------

%% Although information about the occurrence of specimen in regions is turning massive and freely available, users of such data must be aware of some inherent caveats, coming from how the context in which data was recorded. Not all questions might be properly answered with such data, as there might be taxonomic, spatial,... biases.

%% As it is common practice for botanists to record each species once during field work, some important ecological attributes such as the species abundance are not to be directly inferred from such data. {check van Gemerden 2005, from Haripersaud2009}




% ==========================
% ==========================
% Types of biodiversity data
% --------------------------

% Survey data (species checklists) -> examples: Catalogue of life



% ===============
% Occurrence data
% ---------------

%% Characteristics of occurrence data?
%%   - Main assets of a occurrence record: taxon, location, datetime {Graham2004} -> for us, also collectors
%%     

%% What are applications of occurrence data?
%%   - Species Distribution Modeling
%%   - Discovery of new species (check Kemp2015 - Museums: The endangered dead. )

%% Sources of occurrence data?
%%   - Biological collections / Museums;
%%   - Crowdsourcing/citizen science projects;

%% Limitations and caveats
%%   - Biases: collector bias, taxonomic bias, geographic bias...
%%   - {Graham2004 box3}
%%%% Kadmon, R., O. Farber, and A. Danin. 2004. Effect of roadside bias on the accuracy of predictive maps produced by bioclimatic models

%% Presence/absence data
%%   - {Graham2004 box3}


% ================
% References
% ----------------
%% Museum-based informatics{Graham2004}






\section{Data Quality}

\section{Preparing data}
\subsection{Data Selection}
\subsection{Data Cleaning}
\subsection{The Entity Resolution problem}
