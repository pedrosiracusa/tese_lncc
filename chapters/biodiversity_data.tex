\chapter{Understanding Biodiversity Occurrence Data}\label{biodiversity_data}

% ==============
% Herbarium data
% --------------
% Information 



% ==========================
% ==========================
% Types of biodiversity data
% --------------------------

% Survey data (species checklists) -> examples: Catalogue of life



% ===============
% Occurrence data
% ---------------

%% Characteristics of occurrence data?
%%   - Main assets of a occurrence record: taxon, location, datetime {Graham2004} -> for us, also collectors
%%     

%% What are applications of occurrence data?
%%   - Species Distribution Modeling
%%   - Discovery of new species (check Kemp2015 - Museums: The endangered dead. )

%% Sources of occurrence data?
%%   - Biological collections / Museums;
%%   - Crowdsourcing/citizen science projects;

%% Limitations and caveats
%%   - Biases: collector bias, taxonomic bias, geographic bias...
%%   - {Graham2004 box3}

%% Presence/absence data
%%   - {Graham2004 box3}


% ================
% References
% ----------------
%% Museum-based informatics{Graham2004}






\section{Data Quality}

\section{Preparing data}
\subsection{Data Selection}
\subsection{Data Cleaning}
\subsection{The Entity Resolution problem}