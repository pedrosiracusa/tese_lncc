\chapter{Conclusion and Future perspectives}\label{conclusion_perspectives}


% ===========
% Conclusions
% -----------
In this dissertation we proposed the conceptual basis of a new approach for describing the assemblage of biological collections as a social process, driven by the taxonomic interests of contributor collectors as well as their social interactions.
We provided methods for structuring species occurrence data from biological collections into two main classes of network models, each giving distinct perspectives on the recording behavior of collectors.
\textbf{Species-Collector Networks} (SCNs) model interest relations between collectors and taxa they record, whilst \textbf{Collector CoWorking Networks} (CWNs) represent collaborative ties between collectors co-authoring specimens records. 
We use the social network analytics framework \cite{Barbier2011, Stork2015} as the theoretical foundation for characterizing topological features of the networks and uncovering hidden mechanisms leading to the composition of biological collections.
%
% Why structure collections data as networks?
According to \cite{Marin2011}, adopting network-based approaches allows analysts to
(\textit{i}) investigate the effects of interactions between individuals on shaping their own behaviors, rather than simply comparing static attributes of individuals within a population; and
(\textit{ii}) investigate the formation of non-homogeneous communities, composed by individuals interacting with their groups at varying levels of commitment.
These aspects are particularly relevant for biological collections, as collectors are...
%Groups and communities % Stork2015 chapt3
%Discoverying communities is very relevant as it allows us to cluster and summarize collectors with similar profiles.
%We can also identify collectors who are most relevant in each community, and which ones bridge communities.
%Position of the actors % Stork2015 chapt3

% The case study
We demonstrated the use of our models by exploring the dataset from the University of Brasília Herbarium.
In that case study we explored structural properties of both network models and investigated the formation of communities of collaboration and interests.
We also assessed the distinctiveness of collectors regarding their taxonomic interests, their colaborativity with other collectors from within and outside their collaboration and interests communities, and the temporal evolution of collaborative recording in the herbarium.


We also investigated how strongly associated taxonomic groups are, based on their
The model had a relatively good quality because, as the herbarium staff uses the BRAHMS system for data management, giving a quality, consistency to the field containing collectors names.
One possible limitation is the low quality of the collectors field.




% Quality of the recordedBy field
In the scope of this work we assumed the species identity determination issue to be resolved, and handled by the data platform we used, GBIF. 
This should also be considered if one is to use data from other sources, from which the determination issue may not be properly resolved.
The quality of the models are essentially limited by the quality of the field containing collectors names.
As this field is traditionally not very relevant for most biodiversity data applications, it is not surprising that it is relatively overlooked.
Some of the most common issues with the field are naming inconsistencies and authorship omission (described in chapter 2).



% Regarding the UB case study
% --------------------------

We chose the University of Brasília herbarium for our case study due to the relatively high quality of those fields.

%% Collector bias
We observed a strong collector bias, with few collectors responsible for a very large number of species and, conversely, many collectors recording very few species.

Common species tend to be underrepresented in biological collections.
Experienced collectors and specialists are more inclined towards recording rare species while ignoring those that are more common \cite{Nelson1990}.

%% Taxonomic bias
Collectors tend to focus on taxa that are of their direct interest and in specimens that are more conspicuous (\textit{e.g.} flowering plants), overlooking other taxa \cite{VanGemerden2005}.




In a study, authors have built a map of collecting density using occurrence records from the INPA herbarium, and identified regions where most of records were concentrated, which they called the collection centres \cite{Nelson1990}.


It is also important to analyze collectors communities belonging to other collections, and eventually join perspectives from distinct collections to produce a more holistic view of collectors activities.


% One issue with the species bag is that it does not incorporate phylogenetic proximity of the taxa when computing the distance of two collectors.
% It would be nice to compute collectors proximity based not only on the absolute composition of their bags but also on the phylogenetic distance of taxa themselves.
% Look at the word2vec algorithm... may be there's some perspective there..


% ============
% Perspectives
% ------------

Although we have provided a basis for the network-based framework, we present some features which we believe are essential to be incorporated in order to allow broader applications of the model.

% Q1: Influential analysis
\paragraph*{How does a influential collectors influence others?}
In the context of our work, we could investigate how the collecting behavior and taxonomic interests of novice collectors are influenced by their association with more experienced ones, at least at the beginning of their careers.
As it naturally happens in many social systems, actions taken by individuals can be strongly influentiated by the behavior of others at more privileged positions.
The influential power of an individual depends not only on the absolute number of connections they hold with other members, but also on how strongly they intermediate other connections, how close they are to every other individual in the network, and how influential are their own connections.

% Q2: Structural equivalence %  see Stork2015 ch3
Nodes are structurally equivalent if they occupy similar positions in the network, with similar connections.
Is a simplistic measure of the role of the node
Collectors in SCNs are equivalent if they collect similar 
structural equivalent collectors are redundant
Taxa recorded by collectors who do not have structural equivalence are more succeptible to stop being recorded, in the absence of its collector.


% Entities resolution and joining with other botanist datasets
% ------------------------------------------------------------
% We want to validate the ids used for collectors in the occurrences dataset, using the structure of the CWN as a starting point. 
% We select a subset of most influential collectors (and we manually resolve their identities) and ask them to identify and forward a message to 5 of their acquainted we suggest. The acquaintances are suggested based on the strengtheness of their links in the CWN.
% We must discover a set of nodes which would make the messages flow more efficiently, reducing the chance of being ignored.
% If the messages are directed from a more influential collector to a less influential one, we expect it to be less likely to be answered and forwarded.
% The CWN provides a structure of collaborations, such that botanists who have collaborated in field are expected to be acquainted to each other.
% We can identify key collectors from which information spreads most efficiently, and message them asking them to identify their collaborators, pointing out more structured references on the web, such as in the lattes platform, orcid....
% Stork2015 chapt.2 -> transmission 

% Including the temporal and geographic dimensions
% ------------------------------------------------
% Including the temporal dimension
% Social network are by nature dynamic, as they evolve over time. New interactions between entities form continuously whereas other ties break.
% Observers' perception is liable to change over time, being influenced by factors such as his/her own interests, motivations, age in career, available resources and location of residence.
% In our study, however, we haven't yet included the temporal dimension. 

% Including the geographic dimension
% The study of collectors carreer trajectories cite{Borgatti2015, conclusion} depends on incorporating temporal and geographic dimensions.


% Adopting an unifying structure
% SCNs and CWNs should be stored in an unified structure, for example using the structure of multiaspect graphs (MAGs).
% MAGs allow representing edges composed of multiple features by using general graph theory, instead of tensor algebra.
% Important dimensions can be included as aspects, such as temporal and spatial, and different types of edges. 


% Identification of homonymous collectors
% ---------------------------------------
% We could use the CWN structure as one additional resource for screening possible homonymous collectors: ?? think about it
% Collectors that are the only link between two communities are candidates, especially when they link to many collectors in each of those communities. 
% In our CWN, for example, carvalho,avm is known to be a homonymous at least for for Antônio Mendes de Carvalho (bus driver and field assistant at UB) and André Maurício de Carvalho (a well known collector from Bahia) (Carolyn personal communication).



% ============
% Applications
% ------------
Here we briefly discuss some of the possible further developments from our models.



% Ellaboration of red lists
% -------------------------
% Red lists are necessary for pointing prioritary species for conservation e.g. the Brazilian flora red list
% Besides data, it is also necessary to include a team of specialists for validating and evaluating the conservation status of species.
% We can use interest networks for systematically selecting potential collaborators
% We can use the expertise of collectors towards the areas that they have visited and the distributions of species of their expertise;
% We also can identify taxonomist specialists as those who determine the taxonomic identity of exsiccates (species-identifier networks, extending CWNs). They would best contribute as taxonomists
% Then we can select an optimal set of specialists for contributing in more specific aspects of the elaboration of lists

% Species-determiners networks
% Also model species-identifiers network, where people are linked to species they identify.
% We could additionally model networks of species and determiners, similarly to what we've present for species and collectors.


% Collectors Profiling and activities history
% -------------------------------------------
% Profiling collectors in terms of their activities and interests can be a way of further detecting anomalies (activity monitoring, Fawcett and Provost 1999).
% Collectors temporal, geographic and taxonomic ranges \cite{}.


% Building Recommender Systems
% ----------------------------
\paragraph*{Building recommender systems.}
% Sampling Site Recommendation
% Assisted planning of future biodiversity surveying is key for improving herbarium data \cite{Graham2004}.
% Strategic sampling in unsurveyed areas: identify gaps in environmental and geographic coverages \cite{Graham2004}.
% Objectivelly priorizing regions and taxa for surveys cite{Graham2004}, site selection cite{Funk2002}
% Sampling site priorization may be done based on niche models {Raxworthy}
% GDMs {Ferrier: Mapping Spatial Pattern in Biodiversity for Regional Conservation Planning: Where to from Here?}

% Collaborative filtering: The system gather information about the interest of the collectors and then proposes collectors to record new species based on the interests of others.
% Team formation support: How to optimally assemble a team of specialists who are willing to work together?
% Link prediction: Trying to predict which ties are most likely to form between entities in the network in the near future.

% Use Case: "From your collection activity pattern, you might be interested in collecting groups {} in places {}...we found a gap there. Why don't you contact team {}? They have extensively collected other groups in that location and are willing to collaborate in field. Otherwise you could contact land owner {}. His property is within that are and his renting fee is {} reais."
% < add illustration >


% Collector's productivity Score
% ------------------------------
\paragraph*{Assessing collectors productivity.}
On their duty of managing the application of public resources to scientific initiatives, research funding agencies deal with the problem of prioritizing proposals from researchers who are most capable of prividing signifficant advance in their respective fields.
Metrics for assessing the academic productivity of researchers usually take into account metrics such as the number of papers published in high-impact journals. 
In this sense, the work of field naturalists has been largely unappreciated by financing agencies.
As a consequence, researches have directed their careers towards activities that are
Another possible application of the network models is assigning scores to collectors based on their recording patterns, which could become produtivity metrics.
Those metrics can be used by science financing agencies, to incentivate scientists to invest in their careers as naturalists, aggregate scientific value to it.
Their metrics depends on centrality scores.
As the metrics are calculated based on published occurrence records, this would incentivate data curators towards sharing their data, supporting open science.


% SDM
% ---
% Background data selection
% Which records can we use as background data? 
% Pseudo-absences are more likely if a group of collectors potentially intereste in the species (or taxonomic group) have searched the area.







% Extending to other types of biodiversity data
% ---------------------------------------------
% Although here we have tackled the specific class of collection (herbarium), this framework can be extended to others communities, such as citizen scientists, photographers...
