\chapter{Conclusion and Future perspectives}\label{conclusion_perspectives}


% ===========
% Conclusions
% -----------



We show that modeling biological collections formation by focusing on the interests and relationships between collectors can be insightful for uncovering hidden mechanisms leading to the current composition of the herbarium. 


% An issue with species bag 
% One issue with the species bag is that it does not incorporate phylogenetic proximity of the taxa when computing the distance of two collectors.
% It would be nice to compute collectors proximity based not only on the absolute composition of their bags but also on the phylogenetic distance of taxa themselves.
% Look at the word2vec algorithm... may be there's some perspective there..




% Although we have here explored the UB herbarium, we could also analyze collectors communities belonging to other collections.
% Model quality is limited by the quality of the collectors field, which is usually overlooked in most uses of collections data.

% ============
% Perspectives
% ------------

% Species-determiners networks
% Also model species-identifiers network, where people are linked to species they identify.
% We could additionally model networks of species and determiners, similarly to what we've present for species and collectors.

% Including the temporal dimension
% Social network are by nature dynamic, as they evolve over time. New interactions between entities form continuously whereas other ties break.
% Observers' perception is liable to change over time, being influenced by factors such as his/her own interests, motivations, age in career, available resources and location of residence.
% In our study, however, we haven't yet included the temporal dimension. 


% Including the spatial dimension
% The study of collectors carreer trajectories cite{Borgatti2015, conclusion} depends on incorporating temporal and geographic dimensions.


% Adopting an unifying structure
% SCNs and CWNs should be stored in an unified structure, for example using the structure of multiaspect graphs (MAGs).
% MAGs allow representing edges composed of multiple features by using general graph theory, instead of tensor algebra.
% Important dimensions can be included as aspects, such as temporal and spatial, and different types of edges. 





% ============
% Applications
% ------------


% Collectors Profiling and activities history
% -------------------------------------------
% Profiling collectors in terms of their activities and interests can be a way of further detecting anomalies (activity monitoring, Fawcett and Provost 1999).


% Building Recommender Systems
% ----------------------------
% Collaborative filtering: The system gather information about the interest of the collectors and then proposes collectors to record new species based on the interests of others.
% Team formation support: How to optimally assemble a team of specialists who are willing to work together?


% Collector's productivity Score
% ------------------------------
% Metrics of collectors could be used by science financing agencies
% More value to careers as naturalists
% Incentivate collectors to share records, supporting open science


% SDM
% ---
% Background data selection
% Which records can we use as background data? 
% Pseudo-absences are more likely if a group of collectors potentially intereste in the species (or taxonomic group) have searched the area.




% Link prediction: Trying to predict which ties are most likely to form between entities in the network in the near future.

% Use Case: "From your collection activity pattern, you might be interested in collecting groups {} in places {}...we found a gap there. Why don't you contact team {}? They have extensively collected other groups in that location and are willing to collaborate in field. Otherwise you could contact land owner {}. His property is within that are and his renting fee is {} reais."
% < add illustration >

% Sampling Site Recommendation
% Objectivelly priorizing regions and taxa for surveys cite{Graham2004}, site selection cite{Funk2002}
% Sampling site priorization may be done based on niche models {Raxworthy}
% GDMs {Ferrier: Mapping Spatial Pattern in Biodiversity for Regional Conservation Planning: Where to from Here?}


