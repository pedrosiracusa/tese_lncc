\chapter{Conclusion and Future perspectives}\label{conclusion_perspectives}


% ===========
% Conclusions
% -----------
We provided a conceptual basis for a network-based framework aiming to describe biological collections as an ensemble of contributions by collectors.
The basis consists, more especifically, of two network models, giving distinct perspectives on the recording behavior of collectors
They can can be structured from species occurrence datasets, largely available for many collections in the world.
The models essentialy depend on two fields: the collectors field and the species name field.
we show that modeling biological collections formation by focusing on the interests and relationships between collectors can be insightful for uncovering hidden mechanisms leading to the current composition of the herbarium. 

% Quality of the recordedBy field
In the scope of this work we assumed the species identity determination issue to be resolved, and handled by the data platform we used, GBIF. 
This should also be considered if one is to use data from other sources, from which the determination issue may not be properly resolved.
The quality of the models are essentially limited by the quality of the field containing collectors names.
As this field is traditionally not very relevant for most biodiversity data applications, it is not surprising that it is relatively overlooked.
Some of the most common issues with the field are naming inconsistencies and authorship omission (described in chapter 2).

We chose the University of Brasília herbarium for our case study due to the relatively high quality of those fields.
It is also important to analyze collectors communities belonging to other collections, and eventually join perspectives from distinct collections to produce a more holistic view of collectors activities.


% One issue with the species bag is that it does not incorporate phylogenetic proximity of the taxa when computing the distance of two collectors.
% It would be nice to compute collectors proximity based not only on the absolute composition of their bags but also on the phylogenetic distance of taxa themselves.
% Look at the word2vec algorithm... may be there's some perspective there..


% ============
% Perspectives
% ------------

Although we have provided a basis for the network-based framework, we present some features which we believe are essential to be incorporated in order to allow broader applications of the model.


% Entities resolution and joining with other botanist datasets
% ------------------------------------------------------------
% We want to validate the ids used for collectors in the occurrences dataset, using the structure of the CWN as a starting point. 
% We select a subset of most influential collectors (and we manually resolve their identities) and ask them to identify and forward a message to 5 of their acquainted we suggest. The acquaintances are suggested based on the strengtheness of their links in the CWN.
% We must discover a set of nodes which would make the messages flow more efficiently, reducing the chance of being ignored.
% If the messages are directed from a more influential collector to a less influential one, we expect it to be less likely to be answered and forwarded.
% The CWN provides a structure of collaborations, such that botanists who have collaborated in field are expected to be acquainted to each other.
% We can identify key collectors from which information spreads most efficiently, and message them asking them to identify their collaborators, pointing out more structured references on the web, such as in the lattes platform, orcid....
% 

% Including the temporal and geographic dimensions
% ------------------------------------------------
% Including the temporal dimension
% Social network are by nature dynamic, as they evolve over time. New interactions between entities form continuously whereas other ties break.
% Observers' perception is liable to change over time, being influenced by factors such as his/her own interests, motivations, age in career, available resources and location of residence.
% In our study, however, we haven't yet included the temporal dimension. 

% Including the geographic dimension
% The study of collectors carreer trajectories cite{Borgatti2015, conclusion} depends on incorporating temporal and geographic dimensions.


% Adopting an unifying structure
% SCNs and CWNs should be stored in an unified structure, for example using the structure of multiaspect graphs (MAGs).
% MAGs allow representing edges composed of multiple features by using general graph theory, instead of tensor algebra.
% Important dimensions can be included as aspects, such as temporal and spatial, and different types of edges. 





% ============
% Applications
% ------------
Here we briefly discuss some of the possible further developments from our models.



% Ellaboration of red lists
% -------------------------
% Red lists are necessary for pointing prioritary species for conservation e.g. the Brazilian flora red list
% Besides data, it is also necessary to include a team of specialists for validating and evaluating the conservation status of species.
% We can use interest networks for systematically selecting potential collaborators
% We can use the expertise of collectors towards the areas that they have visited and the distributions of species of their expertise;
% We also can identify taxonomist specialists as those who determine the taxonomic identity of exsiccates (species-identifier networks, extending CWNs). They would best contribute as taxonomists
% Then we can select an optimal set of specialists for contributing in more specific aspects of the elaboration of lists

% Species-determiners networks
% Also model species-identifiers network, where people are linked to species they identify.
% We could additionally model networks of species and determiners, similarly to what we've present for species and collectors.


% Collectors Profiling and activities history
% -------------------------------------------
% Profiling collectors in terms of their activities and interests can be a way of further detecting anomalies (activity monitoring, Fawcett and Provost 1999).


% Building Recommender Systems
% ----------------------------
% Collaborative filtering: The system gather information about the interest of the collectors and then proposes collectors to record new species based on the interests of others.
% Team formation support: How to optimally assemble a team of specialists who are willing to work together?


% Collector's productivity Score
% ------------------------------
% Metrics of collectors could be used by science financing agencies
% More value to careers as naturalists
% Incentivate collectors to share records, supporting open science


% SDM
% ---
% Background data selection
% Which records can we use as background data? 
% Pseudo-absences are more likely if a group of collectors potentially intereste in the species (or taxonomic group) have searched the area.




% Link prediction: Trying to predict which ties are most likely to form between entities in the network in the near future.

% Use Case: "From your collection activity pattern, you might be interested in collecting groups {} in places {}...we found a gap there. Why don't you contact team {}? They have extensively collected other groups in that location and are willing to collaborate in field. Otherwise you could contact land owner {}. His property is within that are and his renting fee is {} reais."
% < add illustration >

% Sampling Site Recommendation
% Objectivelly priorizing regions and taxa for surveys cite{Graham2004}, site selection cite{Funk2002}
% Sampling site priorization may be done based on niche models {Raxworthy}
% GDMs {Ferrier: Mapping Spatial Pattern in Biodiversity for Regional Conservation Planning: Where to from Here?}


% Extending to other types of biodiversity data
% ---------------------------------------------
% Although here we have tackled the specific class of collection (herbarium), this framework can be extended to others communities, such as citizen scientists, photographers...
