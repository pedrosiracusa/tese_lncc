\chapter{Conclusion and Future perspectives}\label{conclusion_perspectives}


%% CONCUSIONS
% Argue that a natural history museum dataset does not represent the biodiversity in its actuation area. Instead, in its best efforts it reflects the perception of people interested in studying it. An observer's perception itself is dynamic, being influenced by factors such as his/her own interests, motivations, age in career, available resources and residence. In other words, the perception varies temporally. Moreover, an observer's perception is also influenced by the perception of others in their own communities. The ensemble of perceptions of all observers that are part of a NHM, summed up with data limitations and quality issues is what best characterizes the collections.





% Building Recommending Systems
%% Collaborative filtering: The system gather information about the interest of the collectors and then proposes collectors to record new species based on the interests of others.

% Perspectives

% ============================
% Sampling Site Recommendation
% ----------------------------
% Objectivelly priorizing regions and taxa for surveys {Graham2004}, site selection{Funk2002}
%% Sampling site priorization may be done based on niche models {Raxworthy}
%% GDMs {Ferrier: Mapping Spatial Pattern in Biodiversity for Regional Conservation Planning: Where to from Here?}

% Recommender System - Use Cases:
%% "From your collection activity pattern, you might be interested in collecting groups {} in places {}...we found a gap there. Why don't you contact team {}? They have extensively collected other groups in that location and are willing to collaborate in field. Otherwise you could contact land owner {}. His property is within that are and his renting fee is {} reais."