\chapter{Conclusion and Future perspectives}\label{conclusion_perspectives}
% ===========
% Conclusions
% -----------
In this dissertation we proposed the conceptual basis of a new approach for describing the assemblage of biological collections as a social process, driven by the taxonomic interests of contributor collectors as well as their social interactions.
We provided methods for structuring species occurrence data from biological collections into two main classes of network models, each giving distinct perspectives on the recording behavior of collectors.
\textbf{Species-Collector Networks} (SCNs) model interest relations between collectors and taxa they record, whilst \textbf{Collector CoWorking Networks} (CWNs) represent collaborative ties between collectors co-authoring specimens records. 
% UB case study
We demonstrated the use of our models by exploring the species occurrence dataset of the University of Brasília Herbarium.
Using the social network analytics framework \cite{Barbier2011,Stork2015} as a theoretical foundation, we explored structural properties of the networks, 
and investigated the formation of communities of collaboration and common interests.
We also assessed the distinctiveness of collectors regarding their taxonomic interests, their collaborativity with others, and the temporal evolution of collaborative recording in the herbarium.


% Why structure collections data as networks?
We believe our models provide the structural basis for a more realistic understanding on how collector and taxonomic biases arise in biological collections. 
As stated by \citeonline{Marin2011}, network-based approaches allow analysts to
(\textit{i}) investigate the effects of interactions between individuals on shaping their own behaviors, rather than simply comparing static attributes of individuals within a population; and
(\textit{ii}) investigate the formation of non-homogeneous communities, composed by individuals interacting with their groups at varying levels of commitment.
We consider that these two aspects are particularly relevant in the context of biological collections.

First, collectors often start their careers being supervised by one or more collectors.
As it naturally happens in many social systems, the behavior of individuals can be strongly influenced by others at more privileged positions, and this is likely to be the case for collectors communities as well.
A network-based approach would allow us to investigate, for instance, how the collecting behavior and taxonomic interests of novice collectors are shaped by their association with more experienced ones.
Moreover, depending on its position on the network, a collector can interact with multiple groups of collectors at different extents, assuming the role of influencer in some cases while being influenced in others.
The influential power of a collector depends not only on the absolute number of connections it holds with others, but also on how strongly it intermediates other connections, how close it is to every other collector in the network, and how influential are its own connections. % Influential analysis
All these aspects can be assessed using well-known metrics (including degree, betweeness, closeness and eigenvector centralities), and could be used for investigating which collectors are most relevant for shaping the taxonomic composition of the collection. 

Second, although collectors often define their own taxonomic interests and expertises in terms of natural or functional groups of organisms, those are not necessarily the groups that best split the interests of collectors in the dataset of a given collection.
In addition, collectors (even the most specialized ones) are not restricted towards exclusively recording organisms of their expertises, nor they have uniform interest towards all of those organisms.
%
In this context, SCNs provide the structure for discovering groupings of taxa that perform better at characterizing and differentiating collectors based on their taxonomic interests (communities in the species projection); and for investigating associations of collectors to groups of taxa in a non-discrete manner, allowing collectors to be linked to taxa at multiple groups and at different intensities.
%
In fact, while some groups of taxa are more specifically recorded by distinctive communities of specialized collectors, others are recorded by a wider range of collectors, with diverse taxonomic interests.
For instance, the SCN of the UB herbarium suggests that collectors specialized in families \textit{Myrtaceae}, \textit{Poaceae} or \textit{Piperaceae} form communities of interests which are in fact more distinctive, as those families are relatively poorly recorded by collectors who are not specialists in each of them (see Figure \ref{fig:ub_scn_family_projSp_communities}).
In contrast, other families such as \textit{Fabaceae}, \textit{Melastomataceae} and \textit{Rubiaceae} are more widely recorded in the herbarium, and thus do not form a tight community of interest.


% Quality of the model
The quality of our network models strongly depends on the quality of the species occurrence dataset which is used to build them, more specifically on the fields containing the names of the collectors (\textit{recordedBy}, in TDWG standards) and the taxonomic identity of the specimen (\textit{scientificName}, in TDWG standards) of each record.
%
During this study we have explored occurrence datasets from other herbaria other than the UB, including
the RB (at the Rio de Janeiro Botanical Garden) \cite{gbif_rb}, 
the MBML (Mello Leitão Herbarium) \cite{gbif_mbml},
and the Hemilio Goeldi Museum Herbarium \cite{gbif_mpegh}.
However, we decided only to use the dataset from the UB in our case study for its relatively high quality, specifically for the two fields mentioned above.
%
In all other herbaria, the collectors field (\textit{recordedBy}) was particularly problematic.
Our hypothesis is that the low quality of this field is associated to its low value for most uses of species occurrence data, implying that not much effort has been employed by data curators towards improving its quality.
While imprecise taxonomic determinations in the \textit{scientificName} field would also lead to low quality networks, this field is critical for many other applications of occurrence data, and thus improving its quality has been extensively pursued by the biodiversity informatics community.
The most common and impacting issues associated to the collectors field are 
($i$) using inconsistent delimiter characters for separating the names of each collector in a record, leading to many non-atomized names and consequently to the existence of nodes in the network representing more than one collector; 
($ii$) registering collectors names using inconsistent naming conventions, which makes it hard to systematically interpret what are the component parts of a name; 
($iii$) using multiple names variations for a collector, leading to collectors being represented by more than one node in the network ; and 
($iv$) only including the name of the first collector in records (and eventually aggregating all secondary collectors under the name `\textit{et. al}'), which is interpreted as an absence of collaborative ties and thus does not contribute for the formation of edges in CWNs.
Constructing the models from a low-quality dataset can thus introduce several semantic imprecisions. 
% RecordedBy: Two classes of issues: (1) Naming inconsistencies; (2) Authorship omission


% Limitations of the model
Our models as proposed in the scope of this dissertation also have a set of limitations, which should be addressed in the future.
% The networks only reflect the view of a single institution -> joining multiple herbaria datasets
First, as the networks are built in a single step using a single dataset (which is usually provided by a single institution), they only represent a partial view of the real interests and collaborations that collectors have accumulated during their careers.
For obtaining more holistic representations, a mechanism for dynamically joining multiple occurrence datasets --- and eventually other types of data --- should be incorporated to the models.
Although one might argue that multiple datasets can be simply merged before they are passed to the constructors of the models, that would still consist of a one-step construction, requiring the availability of all data at first place.
In addition, if any other dataset were to be incorporated in the future, the model would need to be entirely reconstructed, and all necessary preprocessing routines would have to be re-executed in each one of the previous datasets.
%
We believe that the most challenging aspect of joining multiple datasets would be addressing the entity resolution problem across datasets from multiple sources. A possible solution would be to map entities in each dataset to unique identifiers (such as the ORCID\footnote{\url{https://orcid.org/}} or the id in Lattes platform\footnote{\url{http://lattes.cnpq.br/}}, widely adopted by the scientific community in Brazil) using a crowdsourcing strategy, described later in this chapter.

% The geographic and temporal dimensions











It is also important to analyze collectors communities belonging to other collections, and eventually join perspectives from distinct collections to produce a more holistic view of collectors activities.


% One issue with the species bag is that it does not incorporate phylogenetic proximity of the taxa when computing the distance of two collectors.
% It would be nice to compute collectors proximity based not only on the absolute composition of their bags but also on the phylogenetic distance of taxa themselves.
% Look at the word2vec algorithm... may be there's some perspective there..


% ============
% Perspectives
% ------------

Although we have provided a basis for the network-based framework, we present some features which we believe are essential to be incorporated in order to allow broader applications of the model.



% Q2: Structural equivalence %  see Stork2015 ch3
Nodes are structurally equivalent if they occupy similar positions in the network, with similar connections.
Is a simplistic measure of the role of the node
Collectors in SCNs are equivalent if they collect similar 
structural equivalent collectors are redundant
Taxa recorded by collectors who do not have structural equivalence are more succeptible to stop being recorded, in the absence of its collector.


\paragraph*{Crowdsourcing collectors identities.}
% Entities resolution and joining with other botanist datasets
% ------------------------------------------------------------
% We want to validate the ids used for collectors in the occurrences dataset, using the structure of the CWN as a starting point. 
% We select a subset of most influential collectors (and we manually resolve their identities) and ask them to identify and forward a message to 5 of their acquainted we suggest. The acquaintances are suggested based on the strengtheness of their links in the CWN.
% We must discover a set of nodes which would make the messages flow more efficiently, reducing the chance of being ignored.
% If the messages are directed from a more influential collector to a less influential one, we expect it to be less likely to be answered and forwarded.
% The CWN provides a structure of collaborations, such that botanists who have collaborated in field are expected to be acquainted to each other.
% We can identify key collectors from which information spreads most efficiently, and message them asking them to identify their collaborators, pointing out more structured references on the web, such as in the lattes platform, orcid....
% Stork2015 chapt.2 -> transmission 

% Including the temporal and geographic dimensions
% ------------------------------------------------
% Including the temporal dimension
% Social network are by nature dynamic, as they evolve over time. New interactions between entities form continuously whereas other ties break.
% Observers' perception is liable to change over time, being influenced by factors such as his/her own interests, motivations, age in career, available resources and location of residence.
% In our study, however, we haven't yet included the temporal dimension. 

% Including the geographic dimension
% The study of collectors carreer trajectories cite{Borgatti2015, conclusion} depends on incorporating temporal and geographic dimensions.


% Adopting an unifying structure
% SCNs and CWNs should be stored in an unified structure, for example using the structure of multiaspect graphs (MAGs).
% MAGs allow representing edges composed of multiple features by using general graph theory, instead of tensor algebra.
% Important dimensions can be included as aspects, such as temporal and spatial, and different types of edges. 


% Identification of homonymous collectors
% ---------------------------------------
% We could use the CWN structure as one additional resource for screening possible homonymous collectors: ?? think about it
% Collectors that are the only link between two communities are candidates, especially when they link to many collectors in each of those communities. 
% In our CWN, for example, carvalho,avm is known to be a homonymous at least for for Antônio Mendes de Carvalho (bus driver and field assistant at UB) and André Maurício de Carvalho (a well known collector from Bahia) (Carolyn personal communication).



% ============
% Applications
% ------------
Here we briefly discuss some of the possible further developments from our models.



% Ellaboration of red lists
% -------------------------
% Red lists are necessary for pointing prioritary species for conservation e.g. the Brazilian flora red list
% Besides data, it is also necessary to include a team of specialists for validating and evaluating the conservation status of species.
% We can use interest networks for systematically selecting potential collaborators
% We can use the expertise of collectors towards the areas that they have visited and the distributions of species of their expertise;
% We also can identify taxonomist specialists as those who determine the taxonomic identity of exsiccates (species-identifier networks, extending CWNs). They would best contribute as taxonomists
% Then we can select an optimal set of specialists for contributing in more specific aspects of the elaboration of lists

% Species-determiners networks
% Also model species-identifiers network, where people are linked to species they identify.
% We could additionally model networks of species and determiners, similarly to what we've present for species and collectors.


% Collectors Profiling and activities history
% -------------------------------------------
% Profiling collectors in terms of their activities and interests can be a way of further detecting anomalies (activity monitoring, Fawcett and Provost 1999).
% Collectors temporal, geographic and taxonomic ranges \cite{}.


% Building Recommender Systems
% ----------------------------
\paragraph*{Building recommender systems.}
% Sampling Site Recommendation
% Assisted planning of future biodiversity surveying is key for improving herbarium data \cite{Graham2004}.
% Strategic sampling in unsurveyed areas: identify gaps in environmental and geographic coverages \cite{Graham2004}.
% Objectivelly priorizing regions and taxa for surveys cite{Graham2004}, site selection cite{Funk2002}
% Sampling site priorization may be done based on niche models {Raxworthy}
% GDMs {Ferrier: Mapping Spatial Pattern in Biodiversity for Regional Conservation Planning: Where to from Here?}

% Collaborative filtering: The system gather information about the interest of the collectors and then proposes collectors to record new species based on the interests of others.
% Team formation support: How to optimally assemble a team of specialists who are willing to work together?
% Link prediction: Trying to predict which ties are most likely to form between entities in the network in the near future.

% Use Case: "From your collection activity pattern, you might be interested in collecting groups {} in places {}...we found a gap there. Why don't you contact team {}? They have extensively collected other groups in that location and are willing to collaborate in field. Otherwise you could contact land owner {}. His property is within that are and his renting fee is {} reais."
% < add illustration >


% Collector's productivity Score
% ------------------------------
\paragraph*{Assessing collectors productivity.}
On their duty of managing the application of public resources to scientific initiatives, research funding agencies deal with the problem of prioritizing proposals from researchers who are most capable of prividing signifficant advance in their respective fields.
Metrics for assessing the academic productivity of researchers usually take into account metrics such as the number of papers published in high-impact journals. 
In this sense, the work of field naturalists has been largely unappreciated by financing agencies.
As a consequence, researches have directed their careers towards activities that are
Another possible application of the network models is assigning scores to collectors based on their recording patterns, which could become produtivity metrics.
Those metrics can be used by science financing agencies, to incentivate scientists to invest in their careers as naturalists, aggregate scientific value to it.
Their metrics depends on centrality scores.
As the metrics are calculated based on published occurrence records, this would incentivate data curators towards sharing their data, supporting open science.


% SDM
% ---
% Background data selection
% Which records can we use as background data? 
% Pseudo-absences are more likely if a group of collectors potentially intereste in the species (or taxonomic group) have searched the area.







% Extending to other types of biodiversity data
% ---------------------------------------------
% Although here we have tackled the specific class of collection (herbarium), this framework can be extended to others communities, such as citizen scientists, photographers...






%------
% Ideas
% =====

% Growth forms and habits might also be a feature of preference of collectors Haripersaud2009 pg42

% Collectors behavior
% -------------------
% Collectors do not employ uniform sampling effort towards every organism included in their respective groups of interest.
%For instance, experienced collectors tend to focus on recording rare species during collecting expeditions, while overlooking others that are very common and thus assumed to be already well represented in the collection \cite{TerSteege2011}.
%In the case of plant collectors, they also show a preference towards collecting flowering and fruiting materials \cite{VanGemerden2005}.
%Defining the interests of a collector in such a way is thus imprecise, oversimplifying aspects that make taxa to assume different levels of relevance for each collector.

% Collection Centres
% ------------------
%In a study, authors have built a map of collecting density using occurrence records from the INPA herbarium, and identified regions where most of records were concentrated, which they called the collection centres \cite{Nelson1990}.

