\chapter{Introduction}\label{introduction}

% ----------------------
% E-science and Big Data
% ======================

%% We frequently lack a thorough scientific understanding of natural processes although we still need to take scientifically-informed decisions ('Science for Environmental Applications', 4th paradigm chapt.2)




% ------------------------------
% Citizen Science & Open Science
% ==============================

%% Ecology is becoming a data-intensive science, requiring the use of data-driven approaches \cite{Kelling2009}	

%%% Data life cycle for transforming biodiversity data into knowledge \cite{Michener2012}




% ---
% SDM
% ===

%% Ecological Niche Modeling
%%% Models predict species occurrence based on environmental features
%%% We need to understand which are the most relevant factors driving species' distribution
%%% Species distribution in geographic space is inferred from relevant relationships discovered in environmental space
%%% Explain the term and equivalence to Species Distribution modeling
%%% Do SDMs model potential or realized niche?
%%% Why bother modeling instead of creating maps based on point interpolation?

\section{Current challenges on Species Distribution Modeling}