\chapter{Introduction}\label{introduction}
% TODO: Introduction of Chapman2005 to complement intro

% Threats to biodiversity and public policies
As a side effect of the rapid human population growth over the past century, we currently face an alarming scenario of biodiversity crisis, with species being extinct at rates that by far exceed natural background rates \cite{Ceballos2015}.
Several human activities---most remarkably habitat modification and destruction, the indiscriminate use of fertilizers and pesticides, and the introduction of exotic organism---have been identified as important causes of massive biodiversity loss, besides their direct influence on global climate change~\cite{Wilcove1998}.
The high number of species being extinct over a relatively short period suggests the imminence of a new event of mass extinction (also referred to as the sixth extinction), in a magnitude comparable to the previous ``big five'' mass extinction events: The Ordovician-Silurian, Late Devonian, Permian-Triassic, End Triassic and the Cretaceous-Tertiary, all of which strongly related to the effects of global climatic variations \cite{Wake2008}.
%\ped{Acham que preciso detalhar as causas de cada um?}\art{Soh se for possivel faze-lo muito brevemente (em meia-linha cada), do contrario pode ficar maçante. Porem, se for possivel, acho interessante.}\ped{Acho que resolve apontar que mudanças climáticas foram chave em todos eles. Concordam?} \art{Se for isso, SIM! E ai cria o contraste entre mudancas climaticas historicas da evolucao do planeta (falei besteira?) com a atuacao acao do homem nessa mais recente. Para mim, isso basta.}\ped{ok. vou fechar}
In face of this scenario, understanding how environmental changes---and ultimately human activities---affect natural communities has been a central concern in ecology and biodiversity conservation research.


% Biological collections supporting conservation Pyke2010
% Nualart2017: Brings examples of nhc data uses for conservation
% Graham2004: section: application of spatial analysis of NHC for conservation assessment and planning
% Others: Soberon2004, Peterson2002a
In this context, \textbf{biological collections} stand as invaluable sources of primary biodiversity information, physically storing biologic materials that testify to the existence of living organisms over time and geographic space.
Regarded as important natural history repositories, biological collections have been increasingly used for a multitude of ecological and conservationist investigations, including the description of patterns of geographical distribution of organisms and their response to climate change, the selection of areas of high priority for conservation, the construction of a red list of threatened species, and the study of routes of biological invasion, just to cite a few~\cite{Pyke2010, Nualart2017,Kemp2015,Chapman2005b}. %check Chapman2005: Uses of Primary Species-occurrence data
As many of these initiatives require intensive use of biodiversity data, typically covering wide geographic areas and long periods of time, they would become impracticable without biological collections, due to the high costs associated with collecting new data in field on demand.
Besides, more species have been recently discovered by taxonomists by inspecting unidentified materials at biological collections than by exploring and collecting at new locations~\cite{Kemp2015}.
%
One important limitation, however, is that biological collections provide only a sampled partial view of the actual biological diversity within their actuation regions. Furthermore, applications aiming at investigating wider ecological and biogeographic processes should be able to combine data from multiple biological collections.
%\art{Eu incluiria aqui algo indicando que as biological collections formam portanto a visao amostral (no espaco-tempo) que temos sobre a biodiverside existente. Entendo que consolidar essa ideia de visao amostral da realidade vai ser importante para a dissertacao (sobretudo para um generalista como eu). Isso consolida a importancia das biological collections antes de voce comecar a falar de digitalizacao em seguida e prepara o cenario para falar de vies amostral mais adiante}

% biological collections are being digitized, data-intensive analysis
Recent efforts towards large-scale digitization of biological collections, associated with a gradual shift in the mindset of data curators towards open-science, are leading many institutions to publish and provide open access to their biodiversity datasets.
Data aggregators, such as GBIF\footnote{\url{https://www.gbif.org}}, iDigBio\footnote{\url{https://www.idigbio.org}}, and SpeciesLink\footnote{\url{http://splink.cria.org.br}} are also playing a key role in this scenario by providing a centralized and transparent access to primary biodiversity data from many collections worldwide, through Web-based graphical and programmatic interfaces.
By facilitating biodiversity researchers to consume data from multiple institutions, such initiatives have boosted the scientific investigation of broader and more complex aspects of biodiversity, which would be otherwise infeasible~\cite{James2018, Newbold2015}.
However, simply having access to large amounts of data on species occurrences is not necessarily sufficient for carrying out comprehensive biodiversity studies.
Understanding biodiversity patterns often requires the integration of many distinct types of environmental and biological data, coming from diverse sources, with varying levels of complexity and associated with many caveats.
Biodiversity research is therefore becoming a \textit{data-intensive science} \cite{Kelling2009}, dealing with the main challenge of how to transform massive amounts of heterogeneous raw data, most of which were not collected for any specific purpose, into valuable knowledge.

% Data-intensive science
Tackling this challenge requires an important analytical paradigm shift in the biodiversity research community, with the adoption of \textit{data-driven approaches} for analyzing biological data, in addition to more traditional, \textit{hypothesis-driven} ones \cite{Kelling2009}.
Instead of using data for statistically corroborating or refuting an initial set of hypotheses posed by an investigator (and thus hypothesis-driven), a data-driven approach aims at systematic unraveling hidden patterns from data, eventually leading to insights and to the generation of new domain-specific hypotheses.
%
Moreover, the viability of a data-based endeavor depends on properly dealing with data during multiple stages of its \textit{life cycle}, requiring the wide adoption of guidelines, standards, protocols, and documentation routines.
The use of \textit{scientific workflows} has been of great value in this regards, as they allow researchers not only to organize and document each step of their own progress, but also make it reproducible and shareable~\cite{Kelling2009,Talbert2013a,Reichman2011}.
Failing to observe and meet the requirements in any stage of the data life cycle can lead to a variety of limitations, hampering the use of biodiversity data for many applications.

% Data life cycle
According to \citeonline{Michener2012}, the life cycle of ecological data is composed of eight steps: 
($i$)~\textit{data management planning}, in which the researcher outlines how data should be collected, stored, and shared;
($ii$) \textit{data collection}, during which one should properly use recording devices and follow protocols in order to avoid the introduction of errors and uncertainties in collected data;
($iii$) \textit{data quality assurance and control}, which involves the definition of standards and mechanisms for preventing and monitoring errors and inconsistencies in datasets;
($iv$) \textit{data description}, which consists of documenting data with metadata;
($v$) \textit{data preservation}, or the storage of data in a properly curated repository;
($vi$) \textit{data discovery}, which is the process of searching for and gathering relevant data for an intended application;
($vii$) \textit{data integration}, in which data from diverse sources domains should be made structurally compatible; and
($viii$) \textit{data analysis}, the process in which information and knowledge on natural phenomena are extracted from data.

% Biodiversity Informatics
The application of information technology for assisting researchers at each stage of the biodiversity data life cycle has been the main concern of the \textbf{Biodiversity~Informatics}~(BI) community, which has undergone a significant expansion over the last two decades \cite{Soberon2004}.
Notable advances have been achieved in many of the stages listed above, although many still pose important unresolved challenges to be addressed within the next decade \cite{Peterson2015}.
%
Among those challenges, issues regarding the \textit{data quality} (DQ) and \textit{fitness} of primary biodiversity data for their intended use have been thoroughly explored by the BI community~\cite{Chapman2005a}, leading to the development of many methods and tools for assisting the process of data cleaning. 
In this context, a conceptual framework for assessing and managing data quality has been recently proposed by \citeonline{KochVeiga2017}, providing a mechanism for improving the collaborative development and sharing of DQ solutions by the BI community.
%
% Data interoperability \Bisby2000.  involves description (DwC)?
Data \textit{interoperability}, which encompasses the complexities of discovering and integrating data from multiple heterogeneous sources and disciplines \cite{Bisby2000}, has also received historical attention, with efforts of groups and organizations towards developing taxonomic backbones~(\textit{e.g.} ITIS\footnote{\url{https://www.itis.gov/}}, Species2000\footnote{\url{http://www.sp2000.org/}}), 
data aggregators (\textit{e.g.} GBIF, SpeciesLink), and 
data standards and vocabularies (TDWG\footnote{\url{http://www.tdwg.org/}} and Darwin Core\footnote{\url{http://rs.tdwg.org/dwc/}}).

% Our motivation is data bias
In this dissertation, we are particularly motivated by the challenge of characterizing \textit{sampling biases} in data, defined as systematic errors that are introduced in data as an effect of not using random sampling designs \cite{Daru2017,Chrisman1991}.
Sampling biases are typically introduced in biodiversity datasets when collectors record specimens in the field in an opportunistic fashion, deploying uneven sampling efforts throughout the studied area and recording preferentially organisms with particular characteristics over others.
As observed by \citeonline{Nelson1990}, most collecting activity in the herbarium of National Institute of Amazonian Research (INPA) were, at that time, clustered around previously postulated endemism centers.
In addition, collectors consider the accessibility of potential sampling sites while selecting them, and thus locations such as roadsides and the proximities of urban centers are often oversampled \cite{Daru2017}, while others that are more remote remain poorly represented.
%
Sampling biases are in fact one of the main limitations of biological collections, and have been observed to strongly impact the overall quality of models in case they fail to account for them \cite{Newbold2010,Araujo2006,Kramer-Schadt2013}.

As biological collections are typically composed of a variety of specimen records, which are collected opportunistically by multiple collectors at distinct locations and in distinct contexts \cite{Daru2017}, they provide no accurate representation of the biological diversity within their actuation areas \cite{Funk1999}.
For instance, common species are usually underrepresented in biological collections \cite{Nelson1990}, eventually with fewer representatives than rare species, which are more thoroughly searched by experienced collectors \cite{TerSteege2011}.
Also, collectors tend to preferentially sample organisms of their direct interests, especially those that are more conspicuous or charismatic, such as large vertebrates or flowering plants \cite{Newbold2010,Graham2004}.
As a result, the taxonomic composition and the temporal and geographic coverage of records in biological collections are strongly biased towards the interests, behavior and activity periods of the main collectors who contribute to them.
Characterizing bias in such datasets would therefore require a systematic analysis of how the complex arrangements of the perceptions, interests and interactions of collectors shape the overall composition of the collections.

%\art{Pedro, tem que desenvolver aqui na intro bem mais o conteudo da dissertacao a partir do paragrafo abaixo. A primeira frase do paragrafo abaixo esta boa para introduzir a dissertacao. A partir dela tem que expandir MUITO, explicando a visao geral da abordagem, o uso de modelos de redes complexas, os insights por tras dos dois modelos propostos, o estudo de caso, etc. Aqui a introducao deve ser expandida e pode o ser bastante. Eh mesmo para ser uma especie de sumario executivo da dissertacao, de modo a revelar todo o conteduo superficialmente, motivando a leitura do restante. Quem acaba de ler a intro deve saber exatamente do que se trata a dissertacao, qual a contribuicao e ter uma visao geral dos resultados alcancados e perspectivas abertas. EH para motivar a leitura do resto para saber os detalhes. Me parece que da para escrecer facil mais 2-3 paginas. O roadmap do restante da dissertacao deve ser separado no fim.}
%\art{Uma primeira sugestao para atender o acima eh pegar o abstract e expandi-lo aqui.}


% Our goal
Within this scope, we propose the first step towards a novel modeling approach, based on \textbf{social network analysis}, for investigating the assemblage of biological collections as a \textit{social process}, resulting from the collecting activities of collectors and their collaborative interactions.
Networks have been used in a wide range of domains for the investigation of complex systems of interacting entities, from studies of the World Wide Web \cite{Albert1999} to ecological interaction webs \cite{Bascompte2007a}.
However, to the best of our knowledge, network analysis has not yet been applied in BI for investigating the assemblage of biological collections.
The most similar study we could find investigates the formation of botanical exchange clubs from the $19th$ and early $20th$ century in Britain and Ireland, in which botanists corresponded with each other by exchanging plant specimens \cite{Groom2014}.
Another recent study uses network analytics to investigate the connectivity and roles of many organizations in the BI landscape, in terms of how they exchange information \cite{Bingham2017}.
Grounded on recent advances in network science theory~\cite{Barabasibook,Newman2010b} and social network analytics~\cite{Barbier2011,Stork2015}, in this dissertation, we introduce two classes of \textit{network models} for structuring collaborative relations involving pairs of collectors; and interest relations involving collectors and species.

\textbf{Species-Collector Networks} (SCNs) are a particular type of interest networks, representing the interests of \textit{collectors} towards the \textit{species} they have recorded in field.
Interest relationships are directly derived from a species occurrence dataset, and necessarily involve a collector and a species.
The strengthness of the ties are given by the number of times the corresponding collector-species associations are observed in the dataset.
Interest relationships are represented in the network model as \textit{edges}, while collectors and species are modeled as \textit{nodes} belonging to distinct sets.
A \textit{bipartite constraint} in this model ensures that all edges necessarily connect nodes from distinct sets, avoiding the introduction of semantic inconsistencies in the model (for instance, a collector cannot collect another collector).
%
From the topology of SCNs, collectors can be characterized in terms of their preferred taxonomic groups and, conversely, species can be systematically characterized in terms of which types of collectors are typically interested on recording them.
Moreover, a multitude of metrics and algorithms from the \textit{network science} domain can be readily applied for extracting insights from the network structure, such as identifying
the most relevant specialist and generalist collectors; 
species that are widely collected and those which are exclusive of particular groups of collectors; 
groups of collectors who have similar taxonomic interests~(\textit{i.e.}, communities of common interests); and 
groups of species that best distinguish the interests of collectors.


\textbf{Collector CoWorking Networks} (CWNs) are a particular type of collaboration networks, structured from \textit{collaboration} (or \textit{coworking}) ties between \textit{collectors} who have collected specimens together in field.
Collaboration relationships are represented as edges in the network model, each of them involving a pair of collectors (represented as nodes of a single type).
As opposed to SCNs, species are not represented in this model.
Ties are extracted from a species occurrence dataset by linking, in a pairwise fashion, all collectors who were included as authors for each record.
The strength of collaboration ties between a pair of collectors is proportional to the number of times they are observed co-authoring records in the dataset.
%
Our justification for CWN models is that as it happens in many social systems, the behavior and interests of collectors may influence and be influenced by those of colleagues with whom they interact.
We consider coworking ties to be good indicatives of the extent to which collectors interact, thus providing the structure for the spread of behaviors and ideas.
The relative influence and roles played by collectors can therefore be assessed from their position in the network, and the formation of \textit{coworking groups} from the topology of CWNs. 

We demonstrate the practical use of our network models by carrying out a case study, using the species occurrence dataset from the University of Brasília Herbarium~(UB), downloaded through the GBIF platform.
Before building the network models, we first briefly explore the taxonomic, geographic, and temporal coverages of the records in the dataset; and then perform a cleaning routine, in order to improve the quality of the resulting networks.
Once the network models are built, we explore their basic topological features and investigate the formation of communities~(interest communities in SCNs and coworking communities in CWNs).
We also investigate the relative relevance of collectors in the herbarium, regarding both their taxonomic contributions and their social positions.

Finally, we believe our network models open new perspectives for research in BI, specifically for applications that rely on data from biological collections. 
With further developments from our work, we expect to provide a mechanism for systematically classifying collectors according to their expertises, their behaviors and their social roles in the collections they contribute to.
This could be achieved by using network-based routines for assigning discrete profiles to collectors (\textit{e.g.} experienced \textit{vs.} novice, specialist \textit{vs.} generalist).
Another perspective is to enrich species occurrence datasets with contextual information, inferred by observing the composition of collectors associated with each record (and their respective profiles).
Moreover, although we have not yet incorporated the temporal and geographical dimensions to the structure of our networks in this work, we believe this would be a fundamental advance, allowing to investigate how collectors interact and which species they record through time and geographic space. 


In order to encourage and facilitate others to analyze SCNs and CWNs from other biological collections, we make  publicly available a \textit{Python} package, developed during this study.
Our package \textit{Caryocar}\footnote{\url{https://github.com/pedrosiracusa/caryocar}} is built on top of the \textit{NetworkX}\footnote{\url{https://networkx.github.io/}} package, and provides classes and methods for building SCNs and CWNs from species occurrence datasets.


% Roadmap
The remainder of this dissertation is organized as follows.
Chapter~\ref{biodiversity_data} is an overview of the structure of species occurrence data (which is used for building our network models), with a brief discussion about aspects of data quality that are most relevant for this work. 
%
In Chapter~\ref{chapter:network_models}, we start by reviewing general concepts from network science, as well as some of the most relevant metrics that have been used for characterizing the topology of our resulting networks. 
Next, we briefly describe the social network analytics framework and exemplify applications of network analysis on the field of biodiversity research.
We conclude the chapter by formally describing both SCN and CWN models.
%
Chapter~\ref{casestudy_ub} is the case study with the UB herbarium, as mentioned above.
%
We conclude our work in Chapter~\ref{conclusion_perspectives} by pointing out directions for further development and new potential perspectives of applications for our network models.



%%% ===========
%%% Other ideas
%%% -----------

% The use of scientific workflows has been of great value in this regards, as they allow researchers not only to organize and document each step of their own progress but also make it reproducible and shareable \cite{Kelling2009,Talbert2013a}.% check Reichman2011 



% Biological collections representativeness
%An observer's perception itself is biased towards more attractive organisms: for instance, botanists tend to focus on flowering materials \cite{VanGemerden2005}.
%Moreover, more experienced collectors tend to record species that are rare than common species, and thus common species are usually undersampled in collections \cite{Nelson1990}.
%Moreover, an observer's perception is also influenced by the perception of others in their own communities, with whom the observer interacts. 
%The composition of perceptions of all observers that contribute to a biological collection, summed up with data limitations and quality issues is what best characterizes the composition of collections.



% Mitigate threats
% Advances in this directions are not only fundamental for improving scientific understanding on how diverse environmental shape the geographic distribution of organisms, but also for supporting scientific-informed decision-making, allowing managers to elaborate more effective plans for mitigating the impacts of human activities on natural systems.
% Moreover, not all species are equally susceptible to the same categories of threats, and thus successful species-directed conservation efforts should prioritize those that are more sensitive to threats they experience, besides more genetically and ecologically relevant.
% This involves considering several species-specific aspects, including those related to life-history traits (\textit{e.g.} generation length), the sizes and dynamics of the populations and the extension of their geographical distributions \cite{iucn_categ_crit}.
% Investigating such aspects often requires the gathering of substantial support data usually spanning long periods of time, which in many cases turns out to be impractical due to high associated costs.



% Massive environmental data
%Moreover, many applications of biodiversity data require including multiple environmental features, such as temperature, precipitation, humidity, radiation.
%The development and deployment of large arrays of new-generation environmental sensors allow remotely monitoring the Earth in high resolution, generating large volumes of data \cite{Lehning2009}.



% Citizen science in Biodiversity: Hardisty2013
%Recently, non-professional collectors have engaged into biodiversity data recording a large amount and variety (video,photo,audio) of born-digital biodiversity data
%The cheapening and miniaturization of recording devices, associated to their connectivitiy to the internet has opened the opportunity for them to collaborate with scientific research, while also engaging into scientific endeavor.
%Citizen are regarded as "live sensors".
%We refer the practice of collecting large ammounts of data with the help of groups of people (paid or unpaid) as crowdsourcing.
%Citizen science allow sampling much more extense are without spending resources.
%Citizen science projects involve the non-scientific community for recording biodiversity, and this data is used for supporting many research projects.
%Data from citizen science projects are structured similarly, with a few differences: (i) absence of vouchered specimens; (ii) taxonomic determinations result from a collaborative community, not necessarily composed of specialists. 



% Species Distribution modeling -> after biodiversity informatics
% Newbold2010 Applications and limitations of NHM data to conservation and ecology
%One of the major applications of biodiversity primary data is Species Distribution Modeling (SDM).
%Concerned on identifying the most relevant environmental features driving the distribution of species, Species Distribution modeling (SDM) is one application that often uses biodiversity informatics.
%These models predict the geographical distribution of species based on environmental features, such as temperature, precipitation, terrain declivity, among others.
%Species distribution in geographic space is inferred from relevant relationships discovered in environmental space.
%Species occurrence data is used in SDM for model fitting. 
%SDMs allows us to project maps of probable occurrence of species.
%Thus we can prioritize areas for allocating resources for wildlife conservation (priority areas selection).
%Priority areas selection. %[ check Schulman 2003: Nature conservation in Amazonia: the role of biological theory in reserve delimitation – Conserv- acio´n de la naturaleza en la Amazonı´a]
%Current challenge on Species Distribution Modeling include obtaining a sufficient amount of data with sufficient quality \cite{Araujo2006}.
