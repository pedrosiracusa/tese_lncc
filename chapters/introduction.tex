\chapter{Introduction}\label{introduction}

% ----------------------
% E-science and Big Data
% ======================

%% We frequently lack a thorough scientific understanding of natural processes although we still need to take scientifically-informed decisions ('Science for Environmental Applications', 4th paradigm chapt.2)




% ------------------------------
% Citizen Science & Open Science
% ==============================

%% Ecology is becoming a data-intensive science, requiring the use of data-driven approaches \cite{Kelling2009}	

%%% Data life cycle for transforming biodiversity data into knowledge \cite{Michener2012}




% ---
% SDM
% ===

%% Ecological Niche Modeling
%%% Models predict species occurrence based on environmental features
%%% We need to understand which are the most relevant factors driving species' distribution
%%% Species distribution in geographic space is inferred from relevant relationships discovered in environmental space
%%% Explain the term and equivalence to Species Distribution modeling
%%% Do SDMs model potential or realized niche?
%%% Why bother modeling instead of creating maps based on point interpolation?

%%% Maxent must be corrected for sampling bias: check {Kramer‐Schadt, Stephanie, et al. "The importance of correcting for sampling bias in MaxEnt species distribution models." Diversity and Distributions 19.11 (2013): 1366-1379.}\cite{Kramer-Schadt2013}

\section{Current challenges on Species Distribution Modeling}


%%%%%%%%%%%%%
%% MOTIVATION

%% Premiss: Argue that a natural history museum dataset does not represent the biodiversity in its actuation area. Instead, in its best efforts it reflects the perception of people interested in studying it. An observer's perception itself is dynamic, being influenced by factors such as his/her own interests, motivations, age in career, available resources and residence. In other words, the perception varies temporally. Moreover, an observer's perception is also influenced by the perception of others in their own communities. The ensemble of perceptions of all observers that are part of a NHM, summed up with data limitations and quality issues is what best characterizes the collections.
%% This premiss holds similar for crowdsourcing data (citizen science)


