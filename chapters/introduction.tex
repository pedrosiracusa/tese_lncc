\chapter{Introduction}\label{introduction}
% TODO: Introduction of Chapman2005 to complement intro

% Threats to biodiversity and public policies
As a side effect of global environmental changes associated to the rapid growth of the human population, we currently face an alarming scenario of biodiversity loss, as species are disappearing at rates that by far exceed natural extinction rates \cite{Ceballos2015}.
In fact, extinction rates observed for the last century stand out similarly to the $5$ previous events of mass extinctions (Ordovician-Silurian, Late Devonian, Permian-Triassic, End Triassic, and Cretaceous-tertiary), with strong evidence that the current will turn out to be the $6th$ one \cite{Wake2008}.
Among the main types of known threats to biodiversity are climate change, habitat loss and biological invasion, all of which associated to human activities \cite{Wilcove1998}.


% Mitigate threats
In face of this alarming scenario, understanding how environmental factors affect natural communities has been a central concern in ecology and biodiversity conservation research. % check intro (pg.3) in Pyke2010
Advances in this directions are not only fundamental for improving scientific understanding on the mechanistic aspects of environmental impacts, but also for supporting scientific-informed decision-making, allowing us to elaborate more effective plans for mitigating the impacts of human activities on natural systems.
Moreover, not all species are equally susceptible to the same cateogries of threats, and thus successful species-directed conservation efforts should prioritize those that are more sensitive to threats they experience, besides more genetically and ecologically relevant.
This involves considerating several species-specific aspects, including those related to life-history traits (\textit{e.g.} generation length), the sizes and dynamics of the populations and the extension of their geographical distributions \cite{iucn_categ_crit}.
Investigating such aspects often requires the gattering of substantial support data usually spanning long periods of time, which in many cases turns out to be impractical due to high associated costs.

% Biological collections supporting conservation
% Nualart2017: Brings examples of nhc data uses for conservation
% Graham2004: section: application of spatial analysis of NHC for conservation assessment and planning
% Others: Soberon2004, Peterson2002a
In this context, biological collections stand as invaluable sources of biodiversity information, as they hold large amounts of physical exemplaries that testify the occurrence of organisms in both temporal and geographic spaces.
NHC data has been recently used for supporting a variety of conservation-related initiatives, including describing biodiversity, understanding the effects of environmental changes, selecting prioritary areas for conservation, building red lists of endangered species, and understanding the routes of biological invasion \cite{Nualart2017, kemp2015}. %check Chapman2005: Uses of Primary Species-occurrence data
Also, biological collections have potentialized the discovery of new species, especially after DNA sequencing techniques \cite{Kemp2015}.

% biological collections are being digitized -> increasing amounts of data
Recent efforts towards large-scale digitization of specimen records, associated to a gradual shift in the mindset of data curators towards open-science are causing many institutions to publish and provide open access to their data.
Data aggregators, such as GBIF\footnote{\url{https://www.gbif.org}}, iDigBio\footnote{\url{https://www.idigbio.org}} and SpeciesLink\footnote{\url{http://splink.cria.org.br}} are playing a key role in this scenario by providing centralized and transparent access to primary biodiversity data from many collections worldwide through graphical and programatic interfaces.
By facilitating researchers to consume data from multiple sources, such intiatives have potentialized the scientific investigation of broader and more complex environmental phenomena, which would be otherwise infeasible \cite{James2018}.

% Citizen science in Biodiversity: Hardisty2013
Recently, non-professional collectors have engaged into biodiversity data recording a large amount and variety (video,photo,audio) of born-digital biodiversity data
The cheapening and miniaturization of recording devices, associated to their connectivitiy to the internet has opened the opportunity for them to collaborate with scientific research, while also engaging into scientific endeavor.
Citizen are regarded as "live sensors".
We refer the practice of collecting large ammounts of data with the help of groups of people (paid or unpaid) as crowdsourcing.
Citizen science allow sampling much more extense are without spending resources.
Citizen science projects involve the non-scientific community for recording biodiversity, and this data is used for supporting many research projects.
Data from citizen science projects are structured similarly, with a few differences: (i) absence of vouchered specimens; (ii) taxonomic determinations result from a collaborative community, not necessarily composed of specialists. 

% Massive environmental data
Moreover, many applications of biodiversity data require including multiple environmental features, such as temperature, precipitation, humidity, radiation.
The development and deployment of large arrays of new-generation environmental sensors allow remotely monitoring the Earth in high resolution, generating large volumes of data \cite{Lehning2009}.


% Challenges: Errors, biases, data standards...
As a result, biodiversity researchers are currently dealing with massive amounts of data, data-intensive science.
This comes with a set a challenges, as many data complexities arise.




However, there are many data-quality complexities, that must be addressed before data can be used.
This comes with a set of challenges.

Primary biodiversity data from any collections worldwide is thus becoming widely available.

Data-driven approaches are needed \cite{kelling2009}.

% Big data challenge: Data-intensive analysis
Given the challenges concerning the complexities of biodiversity data, new approaches are needed for
We need new approaches for analyzing data, which is currently freely available in massive amounts.
Ecology is becoming a data-intensive science, and thus requires the adoption of data-driven approaches, in addition to more traditionally hypothesis-based approaches \cite{Kelling2009}.
More traditionally, ecology is more hypothesis-driven, as an investigator states hypotheses a priori from her own knowledge, and expects certain patterns to be observed in data she collects.
A data-driven approach, on the other hand, seeks to unravel relevant patterns from data, which can then be used for generating domain-specific hypotheses.
Data-intensive science gives us the opportunity of studying natural systems, which are inherently complex, more thoroughly.




Data is increasingly being made available
However, we still need to take information from it
% Biodiversity informatics % Current challenges in Hardisty2013
Biodiversity informatics provides an evidence-based approach for understanding some of those aspects, based on data.
The application of information technologies for managing, exploring, analyzing and interpreting primary biodiversity data has led to the rise of the field of Biodiversity Informatics \cite{Soberon2004}.
A central goal is to allow interoperability of scattered biodiversity data, as well as support building knowledge from local systems \cite{Bisby200}.
Typically, primary biodiversity data that is useful for biodiversity informatics studies concern specimens occurrence records.
There are costs associated to collecting new biodiversity data.
We need to be able to use biodiversity data that is already available the best as we can, taking as much information as possible from data that is already available.

% SDM for wildlife management
% Newbold2010
Concerned on identifying the most relevant environmental features driving the distribution of species, Species Distribution modeling (SDM) is one application that often uses biodiversity informatics.
These models predict the geographical distribution of species based on environmental features, such as temperature, precipitation, terrain declivity, among others.
Species distribution in geographic space is inferred from relevant relationships discovered in environmental space.
Species occurrence data is used in SDM for model fitting. 
SDMs allows us to project maps of probable occurrence of species.
Thus we can prioritize areas for allocating resources for wildlife conservation (priority areas selection).
Priority areas selection. %[ check Schulman 2003: Nature conservation in Amazonia: the role of biological theory in reserve delimitation – Conserv- acio´n de la naturaleza en la Amazonı´a]
Current challenge on Species Distribution Modeling include obtaining a sufficient amount of data with sufficient quality \cite{Araujo2006}.






% Knowledge from data
However, simply gathering and publishing large amounts of data per se is not sufficient.
One of the main challenges is how we transform a massive amount of messy data into valuable knowledge.
Data is useless unless it can be used for some purpose.
For instance, in order to be useful data needs to be properly organized and validated before being published.
Data life cycle is used for transforming biodiversity data into knowledge \cite{Michener2012}
Data life cycle includes (i) data management planning; (ii) data collection; (iii) data quality assurance and control (using standards and validation methods); (iv) data description, with metadata; (v) data preservation; (vi) data discovering; (vii) data integration; (viii) data analysis. 




% Big data challenge: Data bias
Another issue is data bias, which arises when records are not obtained from random sampling \cite{Daru2017}
Using biased data can impact the quality of models built on it \cite{Newbold2010}.
Data from biological collections (or museum data) is often biased \cite{Daru2017}.
Biological collection data is typically collected in clustered areas (collection centres) with very high collecting effort, and clusters are scattered \cite{Nelson1990, VanGemerden2005}.
Many algorithms for building SDMs assume data is randomly collected, and are thus particularly sensitive to biases \cite{Araujo2016} .
Maxent must be corrected for sampling bias.% check {Kramer-Schadt, Stephanie, et al. "The importance of correcting for sampling bias in MaxEnt species distribution models." Diversity and Distributions 19.11 (2013): 1366-1379.}\cite{Kramer-Schadt2013}
% Bias needs to be characterized, and accounted for.


% Biological collections representativeness
A biological collection does not necessarily represent well the biodiversity within its actuation regions \cite{Funk1999}.
Instead, in its best efforts it reflects the perception and interests of people who have most contributed to it in terms of specimens records.
An observer's perception itself is biased towards more attractive organisms. 
For instance, botanists tend to focus on flowering materials \cite{VanGemerden2005}.
Moreover, more experienced collectors tend to record species that are rare than common species, and thus common species are usually undersampled in collections \cite{Nelson1990}.
Moreover, an observer's perception is also influenced by the perception of others in their own communities, with whom the observer interacts. 
The composition of perceptions of all observers that contribute to a biological collection, summed up with data limitations and quality issues is what best characterizes the composition of collections.
Data users must be aware of some inherent caveats, coming from how the context in which data was recorded. 
This premiss holds similar for crowdsourcing data (citizen science)


% Our goal
We want to characterize collectors in terms of their taxonomic interests and collaborative interactions with other collectors.
We present two network models for modeling collectors interests and collaborative interactions in a biological collection.
We then expect to infer contextual information for each record based on the team of collectors who have authored it.
Thus we use network analysis for enriching biological collection datasets (Dataset contextual enrichment).
Finally, contextual information on the records can be derived from the models, and be integrated to the datasets as metadata. %find reference on biodiversity primary data and metadata
% < Add illustration >



% Overview of the dissertation
In chapter 2 we provide an overview of biodiversity data
%
In chapter 3 we formally propose two network models as a basis for understanding how the taxonomic interests and collaborative behavior of collectors shape the composition of biological collections.
The models are based on graph theory.
%
In chapter 4 we provide a case study using the dataset of the University of Brasília herbarium.
We structure, where we explore theapply the network models for a case study of a real-world biological collection, the University of Brasília herbarium.
%
In chapter 5 we present future perspectives to be achieved with the models here proposed.




