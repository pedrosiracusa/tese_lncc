\chapter{Introduction}\label{introduction}
% TODO: Introduction of Chapman2005 to complement intro

% Threats to biodiversity and public policies
As a side effect of the rapid human population growth over the past century, we currently face an alarming scenario of biodiversity crisis, with species being extinct at rates that by far exceed natural background rates \cite{Ceballos2015}.
Several human activities have been identified as important causes of massive biodiversity loss, most remarkably habitat modification and destruction, the indiscriminate use of fertilizers and pesticides, and the introduction of exotic organisms \cite{Wilcove1998}.
Moreover, human activities are also considered to have a direct influence on global climate change.
The high number of species being extinct over a relatively short period suggests the imminence of a new event of mass extinction (also referred to as the sixth extinction), in a magnitude that is comparable to the previous ``big five'' \cite{Wake2008}.
In face of this scenario, understanding how environmental changes --- and ultimately human activities --- affect natural communities has been a central concern in ecology and biodiversity conservation research.





% Biological collections supporting conservation Pyke2010
% Nualart2017: Brings examples of nhc data uses for conservation
% Graham2004: section: application of spatial analysis of NHC for conservation assessment and planning
% Others: Soberon2004, Peterson2002a
In this context, \textbf{biological collections} stand as invaluable sources of biodiversity information, having been increasingly used in a multitude of ecological and conservationist investigations \cite{Pyke2010}.
Those include the description of patterns of geographical distribution of organisms and their response to climate change, the selection of areas of high priority for conservation, the construction of red list of threatened species, and the study of routes of biological invasion, just to cite a few \cite{Nualart2017,Kemp2015,Chapman2005b}. %check Chapman2005: Uses of Primary Species-occurrence data
As many of these initiatives require intense use of biodiversity data (covering taxonomic, geographical and temporal spaces), they would be impracticable without biological collections, due to high costs associated to collecting new data in field.
Biological collections have also boosted the discovery and description of new species, as they store a considerable amount of unidentified biological material that have never been inspected by taxonomists \cite{Kemp2015}.


% biological collections are being digitized, data-intensive analysis
Recent efforts towards large-scale digitization of biological collections, associated to a gradual shift in the mindset of data curators towards open-science are leading many institutions to publish and provide open access to their biodiversity datasets.
Data aggregators, such as GBIF\footnote{\url{https://www.gbif.org}}, iDigBio\footnote{\url{https://www.idigbio.org}} and SpeciesLink\footnote{\url{http://splink.cria.org.br}} are also playing a key role in this scenario by providing centralized and transparent access to primary biodiversity data from many collections worldwide, through Web-based graphical and programatic interfaces.
By facilitating researchers to consume data from multiple institutions, such initiatives have potentialized the scientific investigation of broader and more complex aspects of biodiversity, which would be otherwise infeasible \cite{James2018, Newbold2015}.
However, simply having access to large amounts of data on species occurrences is not sufficient for carrying out comprehensive biodiversity studies.
Understanding biodiversity patterns often requires the integration of many distinct types of environmental and biological data, coming from diverse sources, with varying levels of complexity and associated with many caveats.
Biodiversity research is therefore becoming a \textit{data-intensive science} \cite{Kelling2009}, dealing with the main challenge of how to transform massive amounts of heterogeneous raw data, most of which were not collected for any specific purpose, into valuable knowledge.

% Data-intensive science
Tackling this challenge requires an important analytical paradigm shift in the biodiversity research community, with the adoption of \textit{data-based approaches} over more traditional, \textit{hypothesis-based} ones \cite{Kelling2009}.
Instead of using data for statistically corroborating or refuting an initial set of hypotheses posed by an investigator (and thus hypothesis-based), a data-based approach aims at systematic unraveling hidden patterns from data, eventually leading to insights and to the generation of new domain-specific hypotheses.
%
Moreover, the viability of a data-based endeavor depends on properly dealing with data during multiple stages of its \textit{life cycle}, requiring the wide adoption of standards, best practice guidelines, quality control protocols and documentation routines.
Failing to observe and meet the requirements in any stage of the data life cycle can lead to a variety of limitations, hampering the use of data for many applications.

% Data life cycle
According to \citeonline{Michener2012}, the life cycle of ecological data is composed of eight steps, being 
($i$) \textit{data management planning}, in which the researcher outlines how data should be collected, stored and shared;
($ii$) \textit{data collection}, during which one should properly use recording devices and follow protocols in order to avoid the introduction of errors and uncertainties in collected data;
($iii$) \textit{data quality assurance and control}, which involves the definition of standards and mechanisms for preventing and monitoring errors and inconsistencies in datasets;
($iv$) \textit{data description}, which consists of documenting data with metadata;
($v$) \textit{data preservation}, or the storage of data in a properly curated repository;
($vi$) \textit{data discovery}, which is the process of searching for and gathering relevant data for an intended application;
($vii$) \textit{data integration}, in which data from diverse sources domains should be made structurally compatible; and
($viii$) \textit{data analysis}, the process in which information and knowledge on natural phenomena are extracted from data.

% Biodiversity Informatics
The application of information technology for assisting researchers at each stage of biodiversity data life cycle has been the main concern of the \textbf{Biodiversity~Informatics}~(BI) community, which has undergone a significant expansion over the last two decades \cite{Soberon2004}.
Notable advances have been achieved in many of the stages listed above, although many still pose important unresolved challenges to be addressed within the next decade \cite{Peterson2015}.
%
Among those, issues regarding the \textit{quality} (DQ) and \textit{fitness} of primary biodiversity data for their intended use have been thoroughly explored by the BI community \cite{Chapman2005a}, leading to the development of many methods and tools for assisting the process of data cleaning. 
In this context, a conceptual framework for assessing and managing data quality has been recently proposed by \citeonline{KochVeiga2017}, providing a mechanism for improving the collaborative development and sharing of DQ solutions by the BI community.
%
% Data interoperability \Bisby2000.  involves description (DwC)?
%
Data quality and interoperability are particularly relevant for many applications of data from biological collections, many of them based on niche models, widely investigated by the BI community.



% Species Distribution modeling -> after biodiversity informatics
% Newbold2010 Applications and limitations of NHM data to conservation and ecology
Species distribution modeling is a technique for estimating the geographic distribution of organisms based on environmental features.

Concerned on identifying the most relevant environmental features driving the distribution of species, \textit{Species Distribution modeling} (SDM) heavily uses data from biological collections.
These models predict the geographical distribution of species based on environmental features, such as temperature, precipitation, terrain declivity, among others.
Species distribution in geographic space is inferred from relevant relationships discovered in environmental space.
Species occurrence data is used in SDM for model fitting. 
SDMs allows us to project maps of probable occurrence of species.
Thus we can prioritize areas for allocating resources for wildlife conservation (priority areas selection).
Priority areas selection. %[ check Schulman 2003: Nature conservation in Amazonia: the role of biological theory in reserve delimitation – Conserv- acio´n de la naturaleza en la Amazonı´a]
Current challenge on Species Distribution Modeling include obtaining a sufficient amount of data with sufficient quality \cite{Araujo2006}.






% Biodiversity informatics % Current challenges in Hardisty2013
The application of information technology for managing, exploring, analyzing and interpreting primary biodiversity data has led to the rise of the field of \textbf{Biodiversity Informatics} \cite{Soberon2004}.
Biodiversity informatics does not exclusively tackle the consumption of data.
Three broad categories of Biodiversity Informatics are data extraction and capture, data compilation, data display and visualization \cite{Peterson2010}.
A central goal is to allow interoperability of scattered biodiversity data, as well as support building knowledge from local systems \cite{Bisby2000}.
Typically, primary biodiversity data that is useful for biodiversity informatics studies concern specimens occurrence records.
There are costs associated to collecting new biodiversity data.
We need to be able to use biodiversity data that is already available the best as we can, taking as much information as possible from data that is already available.
% Challenges: Errors, biases, data standards..











% Species Distribution modeling -> after biodiversity informatics
% Newbold2010 Applications and limitations of NHM data to conservation and ecology
One of the major applications of biodiversity primary data is Species Distribution Modeling (SDM).
Concerned on identifying the most relevant environmental features driving the distribution of species, Species Distribution modeling (SDM) is one application that often uses biodiversity informatics.
These models predict the geographical distribution of species based on environmental features, such as temperature, precipitation, terrain declivity, among others.
Species distribution in geographic space is inferred from relevant relationships discovered in environmental space.
Species occurrence data is used in SDM for model fitting. 
SDMs allows us to project maps of probable occurrence of species.
Thus we can prioritize areas for allocating resources for wildlife conservation (priority areas selection).
Priority areas selection. %[ check Schulman 2003: Nature conservation in Amazonia: the role of biological theory in reserve delimitation – Conserv- acio´n de la naturaleza en la Amazonı´a]
Current challenge on Species Distribution Modeling include obtaining a sufficient amount of data with sufficient quality \cite{Araujo2006}.










% Big data challenge: Data bias
Another issue is data bias, which arises when records are not obtained from random sampling \cite{Daru2017}
Using biased data can impact the quality of models built on it \cite{Newbold2010}.
Data from biological collections (or museum data) is often biased \cite{Daru2017}.
Biological collection data is typically collected in clustered areas (collection centres) with very high collecting effort, and clusters are scattered \cite{Nelson1990, VanGemerden2005}.
Many algorithms for building SDMs assume data is randomly collected, and are thus particularly sensitive to biases \cite{Araujo2016} .
Maxent must be corrected for sampling bias.% check {Kramer-Schadt, Stephanie, et al. "The importance of correcting for sampling bias in MaxEnt species distribution models." Diversity and Distributions 19.11 (2013): 1366-1379.}\cite{Kramer-Schadt2013}
% Bias needs to be characterized, and accounted for.


% Biological collections representativeness
Biological collections are not faithful representations of the biological diversity within their actuation regions \cite{Funk1999}.
Instead, they result from complex arrangements of perceptions and interests of people who contribute to them.
An observer's perception itself is biased towards more attractive organisms. 
For instance, botanists tend to focus on flowering materials \cite{VanGemerden2005}.
Moreover, more experienced collectors tend to record species that are rare than common species, and thus common species are usually undersampled in collections \cite{Nelson1990}.
Moreover, an observer's perception is also influenced by the perception of others in their own communities, with whom the observer interacts. 
The composition of perceptions of all observers that contribute to a biological collection, summed up with data limitations and quality issues is what best characterizes the composition of collections.
Data users must be aware of some inherent caveats, coming from how the context in which data was recorded. 
This premiss holds similar for crowdsourcing data (citizen science)


% Our goal
In this dissertation we propose a modeling approach for structuring 
We want to characterize collectors in terms of their taxonomic interests and collaborative interactions with other collectors.
We present two network models for modeling collectors interests and collaborative interactions in a biological collection.
We then expect to infer contextual information for each record based on the team of collectors who have authored it.
Thus we use network analysis for enriching biological collection datasets (Dataset contextual enrichment).
Finally, contextual information on the records can be derived from the models, and be integrated to the datasets as metadata. %find reference on biodiversity primary data and metadata
% < Add illustration >



% Overview of the dissertation
In chapter 2 we provide an overview of biodiversity data.
For those already familiarized with biodiversity informatics, we suggest skipping chapter 2.
%
In chapter 3 we formally propose two network models as a basis for understanding how the taxonomic interests and collaborative behavior of collectors shape the composition of biological collections.
The models are based on graph theory.
%
In chapter 4 we provide a case study using the dataset of the University of Brasília herbarium.
We structure, where we explore theapply the network models for a case study of a real-world biological collection, the University of Brasília herbarium.
%
In chapter 5 we present future perspectives to be achieved with the models here proposed.
%
As this text is intended for a multidisciplinary audience, we suggest four different reading strategies, which are listed below.
The $+$ or $-$ signs next to the acronyms \textbf{BI} (Biodiversity Informatics) and \textbf{GT} (Graph Theory) represent the familiarity of the reader with the respective areas.

\paragraph*{Strategy 1: $\text{BI}^-$ , $\text{GT}^-$.} Starting by first looking at the general aspects of graphs and bipartite graphs (Figures \ref{fig:graphs} and \ref{fig:bipartite_general}) and then relate them to the network models we propose (Figures \ref{fig:scn_general}, \ref{fig:cwn_general}), for a general intuition.

\paragraph*{Strategy 2: $\text{BI}^+$ , $\text{GT}^-$.} For those familiarized with biodiversity informatics but not with graph theory we suggest skipping chapter \ref{biodiversity_data} (use it as a reference) and then ha

\paragraph*{Strategy 3: $\text{BI}^-$ , $\text{GT}^+$.} For those familiarized with graph theory but not biodiversity informatics

\paragraph*{Strategy 4: $\text{BI}^+$ , $\text{GT}^+$.} For those familiarized with both graph theory and biodiversity informatics we recommend skipping chapter \ref{biodiversity_data} and section \ref{section:networkscience}.






%%% ===========
%%% Other ideas
%%% -----------

% The use of scientific workflows has been of great value in this regards, as they allow researchers not only to organize and document each step of their own progress but also make it reproducible and shareable \cite{Kelling2009,Talbert2013a}.% check Reichman2011 


% Mitigate threats
% Advances in this directions are not only fundamental for improving scientific understanding on how diverse environmental shape the geographic distribution of organisms, but also for supporting scientific-informed decision-making, allowing managers to elaborate more effective plans for mitigating the impacts of human activities on natural systems.
% Moreover, not all species are equally susceptible to the same categories of threats, and thus successful species-directed conservation efforts should prioritize those that are more sensitive to threats they experience, besides more genetically and ecologically relevant.
% This involves considering several species-specific aspects, including those related to life-history traits (\textit{e.g.} generation length), the sizes and dynamics of the populations and the extension of their geographical distributions \cite{iucn_categ_crit}.
% Investigating such aspects often requires the gathering of substantial support data usually spanning long periods of time, which in many cases turns out to be impractical due to high associated costs.


% Massive environmental data
%Moreover, many applications of biodiversity data require including multiple environmental features, such as temperature, precipitation, humidity, radiation.
%The development and deployment of large arrays of new-generation environmental sensors allow remotely monitoring the Earth in high resolution, generating large volumes of data \cite{Lehning2009}.


% Citizen science in Biodiversity: Hardisty2013
%Recently, non-professional collectors have engaged into biodiversity data recording a large amount and variety (video,photo,audio) of born-digital biodiversity data
%The cheapening and miniaturization of recording devices, associated to their connectivitiy to the internet has opened the opportunity for them to collaborate with scientific research, while also engaging into scientific endeavor.
%Citizen are regarded as "live sensors".
%We refer the practice of collecting large ammounts of data with the help of groups of people (paid or unpaid) as crowdsourcing.
%Citizen science allow sampling much more extense are without spending resources.
%Citizen science projects involve the non-scientific community for recording biodiversity, and this data is used for supporting many research projects.
%Data from citizen science projects are structured similarly, with a few differences: (i) absence of vouchered specimens; (ii) taxonomic determinations result from a collaborative community, not necessarily composed of specialists. 
