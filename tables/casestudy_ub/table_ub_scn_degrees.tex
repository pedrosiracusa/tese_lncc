\caption{Some degree metrics for the UB SCN model. For each nodes set the total number of nodes, average degree $\langle k \rangle$, top-10 highest-degree nodes and their respective degrees $k$ are listed. We define $k^*$ as the maximum possible degree of a nodes set, a metric that represents the degree of a hypothetical node which is connected to every single node from the complementary set. Therefore $k/k^*$ is the proportion of nodes from the complementary set a given node is linked to.}
  \begin{center}
  \begin{tabular}{l c c c c c}
    & num of nodes & $\langle k \rangle$ & top-10 & $k$ & $k/k^*$ \\
   \hline    collectors & 6768 & 21.08 &
   \begin{tabular}[t]{{@{}c@{}@{}}}irwin,hs\\heringer,ep\\anderson,wr\\proenca,ceb\\ratter,ja\\faria,jeq\\eiten,g\\souza,rr\\harley,rm\\santos,rrb\end{tabular} &
   \begin{tabular}[t]{{@{}c@{}@{}}}4535\\2586\\2156\\1888\\1803\\1681\\1586\\1549\\1514\\1502\end{tabular} &
   \begin{tabular}[t]{{@{}c@{}@{}}}0.30\\0.17\\0.14\\0.12\\0.12\\0.11\\0.10\\0.10\\0.10\\0.10\end{tabular} \\ \\
    species & 15344 & 9.30 &
   \begin{tabular}[t]{{@{}c@{}@{}}}\textit{Myrcia splendens}\\\textit{Myrcia guianensis}\\\textit{Eugenia punicifolia}\\\textit{Casearia sylvestris}\\\textit{Palicourea rigida}\\\textit{Myrcia tomentosa}\\\textit{Qualea parviflora}\\\textit{Solanum lycocarpum}\\\textit{Piper aduncum}\\\textit{Miconia albicans}\end{tabular} &
   \begin{tabular}[t]{{@{}c@{}@{}}}388\\335\\266\\258\\241\\239\\232\\228\\209\\201\end{tabular} &
   \begin{tabular}[t]{{@{}c@{}@{}}}0.06\\0.05\\0.04\\0.04\\0.04\\0.04\\0.03\\0.03\\0.03\\0.03\end{tabular} \\ 
  \hline
  \end{tabular}
  \end{center}
  \label{table:ub_scn_degrees}
